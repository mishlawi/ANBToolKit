\documentclass{article}
\usepackage[utf8]{inputenc}
\usepackage{imakeidx}
\makeindex
\title{Industrias de Adelino Vilar}
\author{Manuel Vilar}
\date{19 de Março de 2017}
\begin{document}

\maketitle
Em Terroso, nas primeiras décadas do século, o meu pai, Adelino Gonçalves Vilar, investiu em pequenas empresas, quase artesanais, dos mais variados artigos, que na altura empregava alguns homens e mulheresque não tinham outro modo de subsistência.
Para ir substitutindo os moinhos mais artesanai, fez uma fábrica de moagem, que com duas pedras moía os cereais dos lavradores.
No lugar do Vilar, onde residia com a numerosa familia que constituiu, também teve uma pequena fábrica/indústria de desnatar o leite das vacas para fazer a manteiga, que era vendida em grandes caixas de madeira.
Mais tarde montou uma fábrica de refirgerntas que fabricava laranjadas, licores e também os chamados `pirolitos`.
Além disso, criiou uma montagem de dois engenhos para descascar linho e uma fábrica de fazer papel e cartir a partir da reciclagem desses produtos, 
Para completar a lista recorde-se a indústria de produzir tacões em madeira para calçado de homem e senhora.
Assim, em meados do século XX, Terroso foi talvez a freguesia com mais empreendedorismo e indústrias do concelho da Póvoa de Varzim, no lugar do Vilar, onde residia um Homem com o mesmo nome, e que foi o grande responsável por essas obras dando, na altura grande nome à sua Terra e ao seu desenvolvimento.

\printindex
\end{document}
