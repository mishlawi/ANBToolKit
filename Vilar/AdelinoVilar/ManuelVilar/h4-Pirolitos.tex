\documentclass{article}
\usepackage[utf8]{inputenc}
\usepackage{imakeidx}
\makeindex
\title{Pirolitos}
\author{ManuelVilar}
\date{2023-01-05}
\begin{document}

\maketitle

% Here starts the story
A fábrica de refrigerantes do \ind{avô Vilar} foi construída pelo seu pai, \ind{Adelino Vilar}, e por ele herdada.
As laranjadas foram, durante muitos anos, distribuidas num formato que se denominava de \ind{pirolitos}. Hoje em dia utilizam-se especialmente latas. Apesar de não tão longinquo, naquela época usavam-se garrafas de vidro popularmente denominadas de pirolitos. Estas têm uma fisionomia similar a qualquer outra garrafa de vidro, com a exceção do gargalo ter em si imbutida uma bolinha de vidro solta. Qual o objetivo? 
Com a gasosa a ser inserida na garrafa, a bolinha não tinha qualquer outra opção que não ocupar o buraco do gargalo, selando a garrafa. As mesmas eram depois recolhidas por funcionarios da empresa na rua, para as reutilizar
A meio/parte final do governo de Salazar, estas garrafas foram abolidas por problemas de higiene. 
\printindex
\end{document}
