
\documentclass{article}

\usepackage{subcaption}
\usepackage{graphicx}
\usepackage{hyperref}

\title{ Livro dos antepassados }
\author{  }
\date{}

\begin{document}

\maketitle
\tableofcontents

\newpage
\newcounter{tablecounter}


\newpage
\section{Roupa a secar}


    
        \textit{by} False
    


 
    \fbox{
        \begin{minipage}{0.9\textwidth}
            \vspace{0.2cm}
            \textbf{\textit{About}}
            \begin{itemize}
                
                    \item \textit{ avô Vilar }
                
            \end{itemize}
            \vspace{0.2cm}
        \end{minipage}
    }
    

    $\ast$~$\ast$~$\ast$  


    \begin{center}
        \begin{minipage}{0.9\textwidth}
            \setlength{\parskip}{0.2cm}
            \setlength{\parindent}{0cm}
            \fontsize{12pt}{14pt}\selectfont
            


Falava-se, antigamente, de algumas coisas associadas a dias especiais.
Mau olhado e coisas do género. As pessoas geriam-se muito pelo que se
falava e o "diabo estava sempre à espreita". Uma das coisas que não se
fazia na nossa casa era pôr roupa a secar no dia da passagem de ano.
Dizia-se que dava azar.

(Escusado será dizer que o meu avô ria-se destes cuidados da minha avó. Não
faziam sentido, dizia ele.)

        \end{minipage}
    \end{center}

    \stepcounter{tablecounter}
    
        \textsuperscript{\hyperref[table:\arabic{tablecounter}]{See metadata here}}
    


\newpage
\section{Pobreza}


    
        \textit{by} Vitalina Moreira Martins
    


 
    \fbox{
        \begin{minipage}{0.9\textwidth}
            \vspace{0.2cm}
            \textbf{\textit{About}}
            \begin{itemize}
                
                    \item \textit{ açaipas }
                
                    \item \textit{ sulipas }
                
            \end{itemize}
            \vspace{0.2cm}
        \end{minipage}
    }
    

    $\ast$~$\ast$~$\ast$  


    \begin{center}
        \begin{minipage}{0.9\textwidth}
            \setlength{\parskip}{0.2cm}
            \setlength{\parindent}{0cm}
            \fontsize{12pt}{14pt}\selectfont
            


Senti talvez que fosse proveitoso acoplar um conjunto de pequenas
histórias da minha avó relativamente à pobreza que a rodeava, na sua
infância e juventude.

Um problema que havia nos campos, e que afligia principalmente aqueles
que eram mais descuidados e deixavam material para a labuta no campo
durante a noite, ou nalgum sitio mais exposto era que pela miséria,
muitos roubavam as de couro que eram usadas no gado para fazer um tipo de chinelos. As
sulipas, segundo ela, eram basicamente umas solas de couro que se
prendiam ao pé, para aqueles que não tinham dinheiro sequer para
calçado.

        \end{minipage}
    \end{center}

    \stepcounter{tablecounter}
    
        \textsuperscript{\hyperref[table:\arabic{tablecounter}]{See metadata here}}
    



\clearpage
\section{Images}

\newcounter{image}


\newpage
\section{Meta-information}
\newcounter{tablecounter2}


    \stepcounter{tablecounter2}
    \begin{table}[ht!]
        \centering
        \begin{tabular}{|c|c|}
            \hline
            
                \textbf{ id } & \textit{ RoupaASecar } \\
                \hline
            
                \textbf{ format } & \textit{ latex } \\
                \hline
            
                \textbf{ type } & \textit{ Story } \\
                \hline
            
                \textbf{ date } & \textit{ 2023-01-7 } \\
                \hline
            
        \end{tabular}
        \caption{A \textbf{ Story }-\textit{ RoupaASecar }} % Add a caption to the table with the current table counter value
        \label{table:\arabic{tablecounter2}} % Use the current table counter value as the label name
    \end{table}

    \stepcounter{tablecounter2}
    \begin{table}[ht!]
        \centering
        \begin{tabular}{|c|c|}
            \hline
            
                \textbf{ id } & \textit{ Pobreza } \\
                \hline
            
                \textbf{ format } & \textit{ latex } \\
                \hline
            
                \textbf{ type } & \textit{ Story } \\
                \hline
            
                \textbf{ date } & \textit{ 2023-01-12 } \\
                \hline
            
        \end{tabular}
        \caption{A \textbf{ Story }-\textit{ Pobreza }} % Add a caption to the table with the current table counter value
        \label{table:\arabic{tablecounter2}} % Use the current table counter value as the label name
    \end{table}


\end{document}
